\documentclass{article}
\usepackage{polski}
\usepackage[utf8]{inputenc}
\usepackage[top=2cm, bottom=2cm, left=2cm, right=2cm]{geometry}
\usepackage{secdot}
\usepackage{scrextend}
\usepackage{booktabs}
\usepackage{array}
\usepackage{ltablex}
\addtokomafont{labelinglabel}{\sffamily}

\title{\vspace{7cm}\LARGE Analiza algorytmów\\semestr 16Z\\Projekt: Skąpa nauczycielka}
\author{\LargeŁukasz Wlazły\\nr albumu: 269365}
\date{}

\begin{document}
	\maketitle
	\pagenumbering{gobble}
	\newpage
	\pagenumbering{arabic}

	\section{Treść zadania}

	Nauczycielka w przedszkolu chce rozdać ciastka dzieciom w swojej grupie. Dzieci siedzą w linii obok siebie (i nie zmieniają pozycji). Każde dziecko ma przypisaną ocene si, i = 1, 2, 3, 4. zgodnie z wynikiem testu umiejętności.
	Nauczycielka chce dac każdemu dziecko co najmniej jedno ciastko. Jeśli dzieci siedzą obok siebie, dziecko z wyższą oceną musi dostac więcej ciastek niż to z niższą oceną. Nauczycielka ma ograniczony budżet, więc chce rozdać jak najmniej ciastek. Zaproponuj algorytm, który zwróci najmniejszą liczbę ciastek, które musi rozdać nauczycielka.


\end{document}
